\chapter{Logic}
\begin{defn}
Proposition is a statement that is either true or false, but not both.
\end{defn}

\section{Logical operations}

\begin{defn}
Definition of $\lnot$. \\ \\
\begin{tabular}{|c|c|}
\hline
$p$ & $\lnot p$  \\
\hline
T & F \\
\hline
F & T \\
\hline
\end{tabular}
\end{defn}

\begin{defn}
Definition of $\land$. \\ \\
\begin{tabular}{|c|c|c|}
\hline
$p$ & $q$ & $p \land q$ \\
\hline
T & T & T \\
\hline
T & F & F \\
\hline
F & T & F \\
\hline
F & F & F \\
\hline
\end{tabular}
\end{defn}

\begin{defn}
Definition of $\lor$. \\ \\
\begin{tabular}{|c|c|c|}
\hline
$p$ & $q$ & $p \lor q$ \\
\hline
T & T & T \\
\hline
T & F & T \\
\hline
F & T & T \\
\hline
F & F & F \\
\hline
\end{tabular}
\end{defn}

\begin{defn}
Definition of $\iff$. \\ \\
\begin{tabular}{|c|c|c|}
\hline
$p$ & $q$ & $p \iff q$ \\
\hline
T & T & T \\
\hline
T & F & F \\
\hline
F & T & F \\
\hline
F & F & T \\
\hline
\end{tabular}
\end{defn}

\begin{defn}
\label{Definition:implies}
Definition of $\implies$.
\begin{align*}
& p \implies q \\
\iff & (\lnot p) \lor q
\end{align*}
\end{defn}

\section{Quantifiers}
\begin{defn}
Universal quantifier is denoted by $\forall$.
\begin{align*}
\forall x, P(x)
\end{align*}
\end{defn}

\begin{defn}
Existential quantifier is denoted by $\exists$.
\begin{align*}
\exists x, P(x)
\end{align*}
\end{defn}

\begin{axm}
\label{Axiom:forall_land_distributive}
\begin{align*}
\forall x, (P(x) \land Q(x)) \iff (\forall x, P(x)) \land (\forall x, Q(x))
\end{align*}
\end{axm}

\begin{axm}
\label{Axiom:De_Morgan_1}
De Morgan's law
\begin{align*}
\lnot (\forall x, P(x)) \iff \exists x, \lnot (P(x))
\end{align*}
\end{axm}

\begin{axm}
\label{Axiom:De_Morgan_2}
De Morgan's law
\begin{align*}
\lnot (\exists x, P(x)) \iff \forall x, \lnot (P(x))
\end{align*}
\end{axm}

\begin{defn}
\label{Definition:uniqueness_quantifier}
Uniqueness quantifier is denoted by $!\exists$.
\begin{align*}
& !\exists x, P(x) \\
\iff & (\exists x, P(x)) \land (\forall x \forall y (P(x) \land P(y) \implies x = y))
\end{align*}
\end{defn}

\section{Proof technique}

\section{Proposition}
Let $P = P(x_1, x_2, \dots, x_n)$.
Let $Q = Q(x_1, x_2, \dots, x_n)$.
etc

\begin{prop}
\label{Proposition:double_negation}
Double negation
\begin{align*}
\lnot (\lnot P) \iff P
\end{align*} 
\end{prop}

\begin{prop}
\label{Proposition:iff_reflexive}
Reflexive property of iff.
\begin{align*}
P \iff P
\end{align*} 
Proof: \\ \\
\begin{tabular}{|c|c|}
\hline
$P$ & $P \iff P$ \\
\hline
T & T \\
\hline
F & T \\
\hline
\end{tabular}
\end{prop}

\begin{prop}
\label{Proposition:iff_symmetric}
Symmetric property of iff.
\begin{align*}
(P \iff Q) \iff (Q \iff P) 
\end{align*} 
\end{prop}

\begin{prop}
\label{Proposition:iff_Transitive}
Transitive property of iff.
\begin{align*}
((P \iff Q) \land (Q \iff R)) \implies (P \iff R)
\end{align*} 
\end{prop}

\begin{prop}
\label{Proposition:De_Morgan_1}
De Morgan's law
\begin{align*}
\lnot (P \land Q) \iff (\lnot P) \lor (\lnot Q) \\
\end{align*}
\end{prop}

\begin{prop}
\label{Proposition:De_Morgan_2}
De Morgan's law
\begin{align*}
\lnot (P \lor Q) \iff (\lnot P) \land (\lnot Q) \\
\end{align*}
\end{prop}

\begin{prop}
\label{Proposition:land_implies_iff}
\begin{align*}
(P \land Q) \implies  (P \iff Q) \\
\end{align*}
\end{prop}

\begin{prop}
\label{Proposition:iff_contraposition}
\begin{align*}
(\lnot P \iff \lnot Q) \iff (P \iff Q) \\
\end{align*}
\end{prop}
